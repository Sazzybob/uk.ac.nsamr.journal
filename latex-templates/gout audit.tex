% SET ARTICLE STYLE
\documentclass[paper=a4,fontsize=11pt,twocolumn]{article}


% PACKAGES

% Layout
\usepackage[english]{babel} % English language/hyphenation
\usepackage{geometry} % For paper size and layout
\usepackage{ragged2e} % For restating fully justified environment
\usepackage{textcomp} % ??? Can't remember...

% Styling
\usepackage[T1]{fontenc} % For fonts
\usepackage{lmodern} % Font for JSAMR
% \usepackage[sfdefault]{AlegreyaSans} % Different type of font packages
\renewcommand{\familydefault}{\sfdefault}
\usepackage{fancyhdr} % For header and footer with odd and even page params
\usepackage{titlesec} % To edit title formats
\usepackage{lastpage} % For Page X of Y
\usepackage{lineno} % Line numbers for review purposes
\usepackage{apacite} % For citations
\usepackage{color} % For colors
\usepackage[svgnames]{xcolor} % Enabling colors by their 'svgnames'

% Extra objects
\usepackage{graphicx} % For nicer figures
\usepackage{booktabs} % For nicer tables
\usepackage{multicol} % More things for tables (and arranging text)
\usepackage{microtype} % For uneven columns
\usepackage{vwcol} % For uneven columns





% DEFINITIONS
\linenumbers
 \geometry{
	 a4paper,
	 top=20mm,
	 right=15mm,
	 left=15mm,
 	 bottom=20mm,
 	 }
\definecolor{brand-primary}{HTML}{1A2530}
\definecolor{brand-secondary}{HTML}{2C5A8F}
\definecolor{brand-logo-dark}{HTML}{58778C}
\definecolor{brand-logo-light}{HTML}{BCD3DB}
\definecolor{brand-midlight}{HTML}{C2C5C3}
\definecolor{brand-light}{HTML}{DADEDB}
\definecolor{brand-lightest}{HTML}{F2F7F3}

\definecolor{brand-body}{HTML}{E8EBEE}



\titleformat*{\section}{\Large\normalfont}
\titleformat*{\subsection}{\large\normalfont}
\titleformat*{\subsubsection}{\large\normalfont}
\titleformat*{\paragraph}{\normalfont}


% HEADERS AND FOOTERS 
\pagestyle{fancy}
\fancyhf{}
\fancyhead[RE,LO]{ARTICLES}
\fancyhead[LE,RO]{\footnotesize\textcolor{brand-secondary}{Running header of this paper}}
\fancyfoot[LE,LO]{\footnotesize{\textcopyright \textcolor{brand-secondary}{Journal of the Student Association of Medical Research}\\
\hspace{4mm}\footnotesize2017 Vol 1 Issue 1 \emph{doi:xx.xxxx/xxx}}}
\fancyfoot[RE,RO]{\footnotesize Page \thepage\ of \pageref{LastPage}}
\renewcommand{\headrulewidth}{2pt}
\renewcommand{\footrulewidth}{1pt}


% MISC
\graphicspath{{figures/}}
\hbadness=100000 % Turn off those dreadful overfull and underful hboxes
\hfuzz=1000pt
\vbadness=100000
\vfuzz=1000pt
\bibliographystyle{apacite}


% MACROS
\newcommand{\horizontalstrip}[1]{\par\noindent\colorbox{shadecolor}
{\parbox{\dimexpr\textwidth-2\fboxsep\relax}{#1}}}



% START OF DOCUMENT
\begin{document}
\pagestyle{fancy}



%%%%%%%% START HEADER MATERIAL
\twocolumn[
		% JSAMR BANNER
                \begin{@twocolumnfalse}	
                		\begin{minipage}{0.95\textwidth}
                    		\setlength{\fboxsep}{0pt}
                    		\fcolorbox{brand-body}{brand-body}{\parbox{1.05\textwidth}{		
					\fcolorbox{brand-body}{brand-body}{\parbox{0.3\textwidth}{	
							\smallskip					
							\includegraphics[width=0.25\textwidth]{logo-wide-jsamr.pdf}
							\smallskip
                                    			}
                                    		} 
					\fcolorbox{brand-body}{brand-body}{\parbox{0.73\textwidth}{		
							\raggedleft{
								\large\textcolor{brand-primary}{Journal of the Student Association of Medical Research}\\
								\smallskip
								\normalsize\textcolor{brand-primary}{Volume 1, Issue 1}
								}
                                    			}
                                    		} 		
                    			}
                    		} 		                    	
                		\end{minipage} 
                	\end{@twocolumnfalse}

		\bigskip

		% TITLE OF ARTICLE
                \begin{@twocolumnfalse}	
                		\begin{minipage}{1.015\textwidth}
                    		\setlength{\fboxsep}{0pt}		
                 		\fcolorbox{white}{white}{\parbox{0.9\textwidth}{					
    					\huge\textcolor{brand-secondary}{An Audit into the Management of Acute Gout Attacks within the past year at Dr Patel and Partners }
					\bigskip					
                    			}
                    		} 
                    		\setlength{\fboxsep}{10pt}
                   		\fcolorbox{white}{white}{\parbox{0.025\textwidth}{					
    					\includegraphics[width=0.045\textwidth]{logo-dna.pdf}
                    			}
                    		} 		
                		\end{minipage} 
                	\end{@twocolumnfalse}

		% AUTHOR(S)
		\Large{}	

		% AFFILLIATIONS(S)
		\setlength{\columnseprule}{1pt}
		\def\columnseprulecolor{\color{brand-secondary}}
		\begin{multicols}{2}
			\small{$^{*\alpha}$affiliation\smallskip\\
				$^{**}$affiliation\smallskip\\
				$^{***}$affiliation}\bigskip\smallskip\\
			\vfill
			\small{$^{\alpha}$Corresponding author: $saneal1@sheffield.ac.uk$}
			\columnbreak

		% DATES	
			\hangindent=0.8cm 
			\small{
				\hspace{8mm}Received: 16$^{th}$ January 2017\smallskip\\
				Revised: 31$^{st}$ January 2017, 10$^{th}$ February 2017\smallskip\\
				Accepted: 11$^{th}$ February 2017\smallskip\\
				Available online: 12$^{th}$ February 2017\bigskip\\
				\vfill
				\hspace{8mm}Keywords: \colorbox{brand-midlight}{\strut keyword 1} \colorbox{brand-midlight}{\strut keyword 2} \colorbox{brand-midlight}{\strut keyword 3}
				}
		\end{multicols}
		\bigskip		

	\begin{@twocolumnfalse}	

		% START ABSTRACT
		\begin{minipage}{0.95\textwidth}
		\begin{centering}
		\setlength{\fboxsep}{10pt}
		\fcolorbox{brand-logo-light}{brand-logo-light}{\parbox{\textwidth}{
			\Large{Abstract}\bigskip\\
			\normalsize
			\textbf{Introduction}
				Lorem ipsum dolor sit amet, consectetuer adipiscing elit. Aenean commodo ligula eget dolor. Aenean massa. Cum sociis natoque penatibus et magnis dis parturient montes, nascetur ridiculus mus. Donec quam felis, ultricies nec, pellentesque eu, pretium quis, sem.\\
			\textbf{Method}
				Lorem ipsum dolor sit amet, consectetuer adipiscing elit. Aenean commodo ligula eget dolor. Aenean massa. Cum sociis natoque penatibus et magnis dis parturient montes, nascetur ridiculus mus. Donec quam felis, ultricies nec, pellentesque eu, pretium quis, sem.\\
			\textbf{Results}
				Lorem ipsum dolor sit amet, consectetuer adipiscing elit. Aenean commodo ligula eget dolor. Aenean massa. Cum sociis natoque penatibus et magnis dis parturient montes, nascetur ridiculus mus. Donec quam felis, ultricies nec, pellentesque eu, pretium quis, sem.\\
			\textbf{Conclusions}
				Lorem ipsum dolor sit amet, consectetuer adipiscing elit. Aenean commodo ligula eget dolor. Aenean massa. Cum sociis natoque penatibus et magnis dis parturient montes, nascetur ridiculus mus. Donec quam felis, ultricies nec, pellentesque eu, pretium quis, sem.\\
			}
		} 
		\end{centering}
		\end{minipage}
		% END ABSTRACT

		
	\end{@twocolumnfalse}
\vspace{8mm}
] 
%%%%%%%% END HEADER MATERIAL




%%%%%%%% START ARTICLE

% INTRODUCTION
\section{Introduction}

Gout is a chronic rheumatological disease characterised by the deposition of monosodium urate crystals within joints due to a pathologically high serum urate level.1 It is a common condition within the Western world and its prevalence is increasing; in part because we are an aging population and the incidence of gout increases with age, but also because the lifestyle factors that increase the risk of developing gout are becoming more prevalent within modern society.2
Purines are part of our dietary intake and are broken down into uric acid to be renally excreted. Uric acid is also a degradation product of cell turnover within the human body. Pathologically high serum urate levels occur when there is an imbalance between the rate of intake or production and the rate of excretion. High concentrations of purines are found in red meat, processed foods, sugar-sweetened soft drinks, foods high in fructose and seafood. Alcohol consumption, particularly beer, increases lactic acid production which inhibits excretion of uric acid through the kidneys. Therefore, people who have high alcohol intake or diets with high levels of these foods are more at risk of developing gout.1-3
Dr Patel and Partners GP practice is a medium sized practice with 13312 patients based in Rotherham, South Yorkshire. According to Public Health England statistics4, the Rotherham population have a higher than average alcohol intake, a higher percentage of obesity and a lower than average level of physical activity. These levels of alcohol intake and obesity suggest that this area of the country is at a higher risk of developing gout. 
Gout is a progressive disease and without proper treatment can lead to structural joint damage causing chronic pain, reduction in activity and reduced quality of life. An acute attack of gout, or a “flare”, is an episode of inflammation of a joint, typically the 1st metatarsal joint, causing severe pain and a clinically erythematous, warm, swollen joint.1 This is due to the presence of urate crystals within the joint and is self-limiting, but use of NSAIDs or colchicine is recommended to reduce inflammation and duration of the flare. There is a link between gout and the presence of other comorbidities such as hypertension, diabetes, dyslipidaemia and renal disease1-3 and therefore the British Society of Rheumatology recommends screening for co-existing diseases upon follow-up from a gout flare.5 Patients experiencing two or more flares of gout within a 12 month period should be commenced on a urate lowering therapy, such as allopurinol, to prevent their disease from becoming progressive and leading to long-term joint damage.5



% METHODS
\section{Methods}
\label{methods}
\subsection{This is a subheading}
I decided to audit the management of any patient registered with the practice who had had an acute attack of gout within the last year. I used the clinical reporting tool in SystemOne to search for all the patients coded with “Gout NOS” within the time period 2nd September 2018 and 2nd September 2019. This search returned 51 patients.
I ordered them by NHS number and then used a random number generator from the internet to select 30 patients from the list. Each patient record was then analysed manually to determine the information required. 
The sample size comprises 58% of the total population of the Dr Patel and Partners GP Practice that had an acute attack of gout within the last year. This is a significant proportion to be auditing and should return results about the management of gout which is applicable to the entire population.


\subsection{Criteria and Standards}
There is no NICE clinical guidance document to guide the management of gout. There is a NICE clinical knowledge summary6 that provides advice and this is based on the recommendations from the British Society of Rheumatology5 (BSR). Studies1,2,7,8 show that gout incidence is increasing but still the management of gout is not always comprehensive and therefore 7 different criteria questions were selected to try to gain an idea of which recommendations GPs were and weren’t adhering to. 
1.	Has the correct first line agent, NSAID (plus PPI) or colchicine, been prescribed? (100%)
The BSR advises that patients should be treated as soon as possible after the onset of an acute attack of gout to limit the duration, and that patients should be prescribed either an NSAID at maximum dose with a PPI for gastroprotection, or colchicine 500µg twice daily to four times daily. The choice of agent should be dependent on other comorbidities, patient choice and clinical judgement.5
Through discussion with Dr Campbell, my GP supervisor, we decided that the first line agent should always be prescribed or advised (if told to buy over the counter) so set the standard at 100%.

2.	Is there any evidence that lifestyle advice has been given to the patient? (100%)
Many studies1-3,9 have shown that diet aspects and alcohol intake are risk factors for developing gout. In addition, eating certain foods or drinking a lot of alcohol can precipitate an acute attack.2 Educating patients on these risk factors can highlight contributory elements and may prevent them from having further attacks and therefore guidelines5-7,10 suggest that lifestyle advice is key for the management of gout.
After discussions, I decided to set the standard as 100% for this; lifestyle advice and patient education is important for keeping gout symptoms under control and even within a 10 minute consultation, a quick explanation or printing off a patient leaflet is simple to do therefore it should be performed every time.

3.	Is there a uric acid level recorded 4-6 weeks after the acute attack? (90%)
Aspiration of the affected joint and microscopy of the synovial fluid for the presence of urate crystals is the gold standard for diagnosis of gout9 but this rarely happens in general practice. Serum testing of uric acid level is an appropriate investigation for determining diagnosis but measurement of this during the acute attack can provide false negatives. All guidance on management of gout recommends performing the serum urate level at follow up at 4-6 weeks.5-7,10
For some patients, for example those that are housebound, it isn’t necessarily feasible to perform a uric acid level. However, the vast majority of patients should be having their uric acid level recorded so that practitioners can manage them correctly even if already on a urate-lowering therapy – it may be that their medication needs titrating up further. Therefore this standard was set at 90%.

4.	Is there a blood pressure measurement recorded 4-6 weeks after the acute attack? (60%)
5.	Is there a HbA1c recorded 4-6 weeks after the acute attack? (60%)
Both hypertension and diabetes lead to an increased risk of cardiovascular disease.11 Gout has been shown by several studies to be linked with several other comorbidities, including the hypertension, diabetes, hyperlipidaemia and chronic kidney disease, as well as a link to an increased risk of cardiovascular disease.1,2,9. No causal link between gout and cardiovascular disease or any comorbidity has been proven by evidence but at the very least, a gout presentation is a good opportunity to screen for relevant comorbidities and the guidance produced by the BSR is to perform a blood pressure check and a HbA1c at the 4-6 week follow up.5  
Amongst the practitioners at Dr Patel and Partners, little was known about the BSR guidelines. However, measurement of BP and HbA1c at follow up is also recommended in the NICE CKS guidance6 and is an example of good practice. It was determined that the standard for these audit criteria should not be set too high as they are not directly linked to the management of an acute gout attack, however they are potentially relevant to the aetiology of chronic gout and so the standard was set at 60%.

6.	If the patient has had ≥2 attacks per year, has allopurinol been discussed? (90%)
Allopurinol is the first line urate-lowering therapy recommended by both BSR and NICE.5,6 As previously stated, chronic gout (defined as a serum uric acid level >408μmol/L and ≥2 acute attacks of gout per year) can lead to irreversible damage to the joints. Therefore urate-lowering therapy should be offered in this instance. As patients have autonomy, not all patients will want to take allopurinol but it should be discussed to reduce the risk. 
There are some patients for whom allopurinol would not be beneficial but for most patients who have had ≥2 attacks per year, allopurinol should be offered therefore the standard was set at 90%.
 

% RESULTS
\section{Results}

\subsection{This is a subheading}

\subsubsection{This is a sub-subheading}

\begin{table}
\begin{centering}
\caption{A two-column table. Tables have captions above rather than below.}\label{tab:one-column}
	\begin{tabular}[\columnwidth]{ccl}\toprule
		Experiment &  Variable 1 &  Variable 2 \\
		\midrule
		1  & X & A \\
		2  & X & A, B \\
		3  & X & C \\
		4  & Y & A \\
		5  & Y & A, B \\
		6  & Y & C \\
		\bottomrule
	\end{tabular}\par
	\medskip
\textit{Notes for this table if required. Tables have captions above rather than below. All values should be rounded to 2 dp. It is preferred that column rules are avoided, and minimal horizontal rules are used.}\end{centering}
\end{table}

\begin{table*}
\begin{centering}
\caption{A two-column table. Tables have captions above rather than below.}\label{tab:two-column}%
	\begin{tabular}[\columnwidth]{ccccccccc}\toprule
	Condition & V1 & V2 & V3 & V4 & V5 & V6 & V7 & V8 \\
	\midrule
	1 & 0.97 & --0.38 & 0.02 & 0.53 & --0.79 & --0.21 & 0.36 & --0.46 \\
	2 & 0.97 & --0.83 & 0.12 & --0.50 & 0.70 & --0.80 & --0.14 & --0.29 \\
	3 & 0.57 & --0.90 & 0.52 & --0.47 & --0.43 & --0.01 & --0.20 & --0.15 \\
	4 & 0.89 & --0.90 & 0.39 & --0.11 & --0.31 & --0.93 & 0.62 & --0.46 \\
	5 & 0.73 & --0.08 & 0.48 & --0.18 & 0.10 & 0.01 & --0.29 & --0.15 \\
	6 & --0.98 & --0.68 & --0.35 & 0.01 & --0.58 & --0.02 & --0.31 & 0.39 \\
	\cmidrule{2-9}
	Mean & 0.00 & --0.00 & 0.00 & --0.00 & 0.00 & --0.00 & --0.00 & --0.00 \\
	\bottomrule
	\end{tabular}\par
	\medskip
\textit{Notes for this table if required. Tables have captions above rather than below. All values should be rounded to 2 dp. It is preferred that column rules are avoided, and minimal horizontal rules are used.}\end{centering}
\end{table*}

Lorem ipsum dolor sit amet, consectetur adipiscing elit. Praesent tempus, tortor vel mattis tristique, est ligula sollicitudin neque, a fermentum dui elit vitae lectus. Integer posuere mi eu nisl malesuada efficitur. Vestibulum ante ipsum primis in faucibus orci luctus et ultrices posuere cubilia Curae; Vestibulum non orci placerat nunc consectetur euismod quis a augue. Nam quis nulla a magna porttitor porttitor. Vivamus eget consectetur leo. Integer blandit lacus ac placerat cursus. Duis elit ante, malesuada facilisis justo a, vehicula semper justo.

\subsection{This is a subheading}

Lorem ipsum dolor sit amet, consectetur adipiscing elit. Praesent tempus, tortor vel mattis tristique, est ligula sollicitudin neque, a fermentum dui elit vitae lectus. Integer posuere mi eu nisl malesuada efficitur. Vestibulum ante ipsum primis in faucibus orci luctus et ultrices posuere cubilia Curae; Vestibulum non orci placerat nunc consectetur euismod quis a augue. Nam quis nulla a magna porttitor porttitor. Vivamus eget consectetur leo. Integer blandit lacus ac placerat cursus. Duis elit ante, malesuada facilisis justo a, vehicula semper justo.

\subsubsection{This is a sub-subheading}
Lorem ipsum dolor sit amet, consectetur adipiscing elit. Praesent tempus, tortor vel mattis tristique, est ligula sollicitudin neque, a fermentum dui elit vitae lectus. Integer posuere mi eu nisl malesuada efficitur. Vestibulum ante ipsum primis in faucibus orci luctus et ultrices posuere cubilia Curae; Vestibulum non orci placerat nunc consectetur euismod quis a augue. Nam quis nulla a magna porttitor porttitor. Vivamus eget consectetur leo. Integer blandit lacus ac placerat cursus. Duis elit ante, malesuada facilisis justo a, vehicula semper justo.

Phasellus consequat luctus lobortis. Curabitur vulputate maximus tortor, quis dapibus tellus commodo eget. Ut hendrerit at risus vel tristique. Morbi faucibus accumsan ornare. Phasellus faucibus eleifend volutpat. Nullam ut tellus nisi. Praesent dolor sapien, pellentesque vel tortor nec, mattis fermentum tortor. Praesent quis fermentum nibh. Suspendisse libero magna, sodales in erat egestas, pharetra vulputate lectus.


% DISCUSSION 
\section{Discussion}
Within a small sample size of 30, just a few patients who haven’t received the correct management can make a lot of difference to the percentages. In answer to criterion 1, about whether first line treatment for acute gout had been given, the answer was “no” in only 2 cases but that meant that the standard of 100% was not met. In one case, the ULT allopurinol was prescribed immediately without checking a uric acid level or prescribing an anti-inflammatory which is not in accordance with best practice. In the other case, no medication advice was recorded but the patient had had previous gout flares so it may be that they were already aware of how to manage. It’s possible that should another 30 patients have been selected from the total sample of 51, the 100% standard for this criterion may have been met.
With regards to criterion 2, about lifestyle advice, the result was considerably lower at 70% but the standard was still set at 100% therefore this seems like quite a poor outcome. It is possible however, that lifestyle advice was given or discussed and just wasn’t explicitly documented within SystemOne. It’s also the case that for some of these patients, the episode of acute gout included in this audit was not their first and therefore they may previously have been given lifestyle advice and therefore it wasn’t warranted again. For this reason, if I was re-auditing, I would set the standard at less than 100%; lifestyle advice is still important therefore a standard of 80% or 90% may have been more appropriate.
Criterion 3 regarding uric acid level measurement had a result of 60% and a standard of 90%. The standard was already set lower than 100% to account for some patients not warranting a uric acid measurement and therefore this standard should have been achievable. There were 2 cases where the uric acid level had been requested but not performed and therefore this is down to patients not booking in for it, rather than a practitioner’s poor management. However, even removing these 2 cases, the result would still have not met the standard and so this reflects non-compliance with the guidelines.5-7,10
Criteria 4 and 5 were analysing whether other cardiovascular risk factors, namely hypertension and diabetes were screened for at follow-up from an acute gout attack using BP and HbA1c measurements. Standards were purposely set relatively low for these but still the results did not meet the standards, at 27% and 23% respectively. This suggests that many practitioners may be unaware of the need to screen for these and that education or adjustments to practice should be made to try to improve this. It may also be that some of the patients are already known hypertension or diabetes sufferers and therefore it wasn’t necessary but it could be argued that even if this is the case, a line in the notes that just says “past medical history noted” or “CVD screen not warranted” would for example show that the practitioner has considered screening for other CVD risk factors and decided it wouldn’t be relevant. 
The fact that the results for criteria 5 and 6 were so low is unlikely to have impacted on patient care; screening for CVD risk factors is not directly linked to managing the gout flare but is an opportunity for doing so and therefore is an example of best practice.
Discussing allopurinol with those who had had ≥2 gout attacks per year, criterion 6, was the only criterion analysed which achieved, even exceeded, the standard. There were only 18 patients included in this criterion rather than 30 however 100% of them had been offered allopurinol so according to this audit, the practice is performing very well in this aspect.



% CONCLUSION
\section{Recommendations}
Dr Patel and Partners GP practice failed to meet the standard in 5 out of 6 of the criteria analysed. It is likely that some of this is down to lack of awareness of the BSR guidelines. 
When presenting at the practice meeting, I detailed the BSR guidelines and gave each practitioner a copy of them thereby improving awareness. I also contacted the template support line for SystemOne to make them aware of the BSR guidelines and asked them to update their template; these were two initial recommendations that I have been able to action. Therefore my final recommendations to the practice are:
*	Improve documentation of whether lifestyle advice has been given 
*	Encourage practitioners to use the templates available within SystemOne as often they are not used and they can help to guide the consultation and provide reminders of how conditions should be managed
*	Re-audit in 12 months to evaluate whether the above recommendations have improved the quality of acute gout management



%%%%%%%% END ARTICLE


\bibliography{References}
1. Dalbeth, N., Merriman, T.R. and Stamp, L.K. Gout. The Lancet. 2016, 388, pp.2039-2052.
2. Robinson, P.C. Gout – An update of aetiology, genetics, co-morbidities and management. Maturitas. 2018, 118, pp.67-73.
3. Singh, J.A., Reddy, S.G. and Kundukulam, J. Risk factors for gout and prevention: A systematic review of the literature. Current Opinions in Rheumatology. 2011, 23(2), pp.192-202.
4. Public Health England. Local Authority Health Profiles: Rotherham. 2017. Available from: https://fingertips.phe.org.uk/profile/health-profiles/data#page/1/gid/1938132696/pat/6/par/E12000003/ati/102/are/E08000018 [Accessed 02 Sept 19]
5. Hui, M., Carr, A., Cameron, S. et al. The British Society for Rheumatology guideline for the management of gout. Rheumatology. 2017, 56, pp. 1056-1059.
6. National Institute for Health and Care Excellence (NICE). Gout Clinical Knowledge Summary. 2018. Available from: https://cks.nice.org.uk/gout#!management [Accessed 02 Sept 19]
7. Richette, P., Doherty, M., Pascual, E., et al. 2016 updated EULAR evidence-based recommendations for the management of gout. Annals of the Rheumatic Diseases. 2017, 76, pp.29-42.
8. Kuo, C-F., Grainge M.J., Mallen, C. et al. Rising burden of gout in the UK but continuing suboptimal management: a nationwide population study. Annals of the Rheumatic Diseases. 2015, 74, pp.661-667. 
9. Sivera, F., Adrés, M. and Quilis, N. Gout: Diagnosis and treatment. Medicina Clinica. 2017, 148(6), pp.271-276.
10. Drugs and Therapeutics Bulletin. Latest guidance on the management of gout. British Medical Journal. 2018, 362, k2893.
11. Murray, C.J.L et al. Global, regional and national comparative risk assessment of 84 behavioural, environmental and occupational and metabolic risks or clusters of risks for 195 countries and territories, 1990-2017: a systematic analysis for the Global Burden of Disease Study 2017. The Lancet. 2018, 392, pp.1923-1994.

\end{document}

